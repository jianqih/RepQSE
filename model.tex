\documentclass[11pt,a4paper]{article}
\usepackage{bookmark}
\usepackage{graphicx,hyperref,amsmath,bm,url,amsthm}
\usepackage[table]{xcolor}
% \usepackage{fullpage}
\usepackage{lastpage}
\usepackage{enumitem}
\usepackage{fancyhdr}
%\usepackage{physics}
% \usepackage{mathrsfs}
\usepackage{mathtools}
\usepackage{wrapfig}
\usepackage{bm}
%\usepackage{fixdif}
\usepackage[square]{natbib}
\usepackage[utf8]{inputenc} % include umlaute

% -------- Define new color ------------------------------
\usepackage{color}  
\definecolor{darkblue}{rgb}{0.15,0,0.37}
\definecolor{darkred}{rgb}{0.35,0,0.08}
\definecolor{mygrey}{rgb}{0.85,0.85,0.85} % make grey for the table
% --------------------------------------------------------

\hypersetup{
colorlinks=true,
hypertexnames=false,
colorlinks,		
linkcolor={blue},
citecolor={blue},
urlcolor={blue},
%ocgcolorlinks, % this option changes the link color to black when the document is printed, for legibility in b/w 
pdfstartview={FitV},
unicode,
breaklinks=true
} 

\usepackage{microtype,todonotes}
\usepackage[australian]{babel} % change date to to European format
\usepackage[a4paper,text={14.5cm, 25.2cm},centering]{geometry} % Change page size
\setlength{\parskip}{1.0ex} % show new paragraphs with a space between lines
\setlength{\parindent}{0em} % get rid of indentation for new paragraph
\clubpenalty = 10000 % prevent "orphans" 
\widowpenalty = 10000 % prevent "widows"

\hypersetup{pdfpagemode=UseNone} % un-comment this if you don't want to show bookmarks open when opening the pdf

\usepackage{mathpazo} % change font to something close to Palatino

\usepackage{longtable, booktabs, tabularx} % for nice tables
\usepackage{caption,fixltx2e}  % for nice tables
\usepackage[flushleft]{threeparttable}  % for nice tables

\newcommand*{\myalign}[2]{\multicolumn{1}{#1}{#2}} % define new command so that we can change alignment by hand in rows
 
% -------- For use in the bibliography -------------------
\usepackage{multicol}
\usepackage{etoolbox}
\usepackage{relsize}
\parskip 1.0ex
\parindent 0pt
\makeatletter
\renewcommand\itemize{%
  \ifnum \@itemdepth >\thr@@\@toodeep\else
    \advance\@itemdepth\@ne
    \edef\@itemitem{labelitem\romannumeral\the\@itemdepth}%
    \list{\csname\@itemitem\endcsname}{%
      \setlength{\itemsep}{1pt}%
      \setlength{\topsep}{1pt}%
      \setlength{\parsep}{1pt}%
      \setlength{\parskip}{1pt}%
    }%
  \fi}
\renewcommand{\enditemize}{\endlist}
\makeatother
\usepackage[onehalfspacing]{setspace} % Line spacing

\usepackage[marginal]{footmisc} % footnotes not indented
\setlength{\footnotemargin}{0.1cm} % set margin for footnotes (so that the number doesn't stick out) 

\usepackage{pdflscape} % for landscape figures

\usepackage[capposition=top]{floatrow} % For notes below figure

% -------- Use the following two lines for making lines grey in tables -------------------
\usepackage{color, colortbl}
\usepackage{multirow}
% ----------------------------------------------------------------------------------------
\linespread{1.0} % 1.2 times the default spacing
\setlength{\topsep}{0pt}
\setlength{\itemsep}{0pt}

% The following defines a footnote without a marker
\newcommand\blfootnote[1]{%
  \begingroup
  \renewcommand\thefootnote{}\footnote{#1}%
  \addtocounter{footnote}{-1}%
  \endgroup
}

% -------- Customize page numbering ----------------------------
\usepackage{fancyhdr}
\usepackage{lastpage}
 \pagestyle{fancy}
\fancyhf{} % get rid of header and footer
 \cfoot{ \footnotesize{ \thepage } }

% \rhead{\small{\textit{Firstname Surname}}}
% \lhead{\small{\textit{PaperTitle}}}
\renewcommand{\headrulewidth}{0pt} % turn off header line
% --------------------------------------------------------------



\newtheorem{defn}{Definition}
\newtheorem{reg}{Rule}
\newtheorem{exer}{Exercise}
\newtheorem{note}{Note}
\newtheorem{expl}{Example}
\newtheorem{pro}{properties}
\newtheorem{asm}{Assumption}
\newtheorem{them}{Theorem}
\newtheorem{rmk}{Remark}
\newtheorem{coro}{Corollary}
\newtheorem{lem}{Lemma}
\newtheorem{pros}{Proposition}
\newcommand{\N}{\mathbb N}
\newcommand{\Z}{\mathbb Z}
\newcommand{\Q}{\mathbb Q}
\newcommand{\T}{\mathbb T}
\newcommand{\R}{\mathbb R}
\newcommand{\C}{\mathbb C}
\newcommand{\F}{\mathbb F}
\newcommand{\B}{\mathbb B}
\newcommand{\X}{\textbf{X}}

\begin{document}
\section{Model}
An economy: a set $N$ regions indexed by $n$. Each region is endowed with an exogenous quality-adjusted supply of land $H_i$. $\bar{L}$ workers. Iceberg cost: $b_{ni} = b_{in}>1= b_{ii}$. 

Preference over consumption $C_n$ and residential land use $h_n$
\begin{equation}
    U_n = \left( \frac{C_n}{\alpha } \right)^{\alpha } \left( \frac{h_n}{1-\alpha } \right)^{1-\alpha },\quad 0<\alpha <1
\end{equation}
The consumption $C_n$ is defined by CES preference:
\begin{equation}
    C_n = \left[ \sum_{i \in N} \int_{0}^{M_i} c_{ni}(j)^\rho d j\right]^{\frac{1}{\rho}},\quad P_n = \left[ \sum_{i \in N} \int_{0}^{M_i} p_{ni}(j)^{1-\sigma } \mathrm{d} j\right]^{\frac{1}{1-\sigma}}\label{eq:pn}
\end{equation}
The production of labor is 
\begin{equation}
  l_i (j ) = F + \frac{x_i(j)}{A_i}\label{eq:labor}
\end{equation}
We can rewrite $$ x_i(j) = (l_i(j)-F)A_i $$ which implies the $MPL = \frac{w_i}{A_i}$ is fixed. 
Profit maximization and zero profit imply: 
\begin{equation}
  p_{ni}(j) = \underbrace{\frac{\sigma }{\sigma -1}}_{\text{markup}} d_{ni} \frac{w_i}{A_i},\label{eq:pnij}
\end{equation}
and equilibrium output of each variety is equal to a constant that depends on location productivity:
\begin{equation}
  x_i (j) = \bar{x}_i = A_i (\sigma -1) F,
\end{equation}
Plug in Eq~\eqref{eq:labor} we have
\begin{equation}
  l_i (j ) =  \bar{l} = \sigma F.
\end{equation}
\paragraph{Equilibrium.} Given the constant equilibrium employment for each variety, labor market clearing implies the total measure of varieties supplied
\begin{equation}
  L_i = \sum_{j}^{} l_i (j) = \bar{l} M_i \Rightarrow M_i = \frac{L_i}{\sigma F}\label{eq:li}
\end{equation}
\paragraph{Price indices and expenditure shares.} 
Using Eq~\eqref{eq:pnij} and labor market clearing Eq~\eqref{eq:li}, one can express the price index dual to the consumption index ~\eqref{eq:pn} as:
\begin{equation}
  P_n = \frac{\sigma }{\sigma -1} \left( \frac{1}{\sigma F} \right)^{\frac{1}{1-\sigma }} \left[ \sum_{i \in N}^{} L_i \left( d_{ni} \frac{w_i}{A_i} \right)^{1-\sigma } \right]^{\frac{1}{1-\sigma }}\label{eq:pni}
\end{equation}
Using the equilibrium prices and labor market clearing, the share of location $n$'s expenditure on goods produced in location $i$ is 
\begin{equation}
  \pi _{ni} = \frac{M_i p_{ni}^{1-\sigma }}{\sum_{k \in N}^{}M_k p_{nk}^{1-\sigma }} = \frac{L_i \left( d_{ni} \frac{w_i}{A_i} \right)^{1-\sigma }}{\sum_{k \in N}^{}L_k \left( d_{nk} \frac{w_k}{A_k} \right)^{1-\sigma }}\label{eq:pi}
\end{equation}
% Since $x(i) = (\sigma -1)FA_i$. 
Further, we can derive the gravity equation for goods trade, where the bilateral trade between locations $n$ and $i$. 
using \eqref{eq:pni} and \eqref{eq:pi} we have 
\begin{equation}
  P_n = \frac{\sigma }{\sigma -1} \left( \frac{L}{\sigma F \pi _{nn}} \right)^{\frac{1}{1-\sigma }} \frac{w_n}{A_n}. \label{eq:pn}
\end{equation}
Trade balance at each location implies that per capita income $v_n$ equal $w_n$ (wage income) plus per capita expenditure on residential land, $(1-\alpha ) v_n$. namely,
\begin{equation}
  v_n L_n = w_n L_n + (1-\alpha ) v_n L_n \Rightarrow v_n 
   = \frac{w_n}{\alpha }. 
\end{equation}
By combining land market clearing condition $L_n h_n = H_n$ with the FOC of consumer problem $r_n H_n = (1-\alpha ) L_n v_n$, 
\begin{equation}
  r_n = \frac{(1-\alpha ) v_n L_n}{H_n} = \frac{1-\alpha }{\alpha } \frac{w_n L_n}{H_n}. \label{eq:rn}
\end{equation}
Population mobility implies that workers receive the same \textit{real income} in all populated locations, thus 
\begin{equation}
  V_n = \frac{v_n}{P_n^\alpha r_n^{1-\alpha }} = \bar{V}
\end{equation}
where the price index = $P_n^\alpha r_n^{1-\alpha }$. 
Using the prices Eq~\eqref{eq:rn} and Eq~\eqref{eq:pn}
Therefore, the location 
\begin{equation}
  \bar{V} = \frac{A_n^\alpha H_n^{1-\alpha } \pi _{nn}^{-\alpha /(\sigma -1)} L_n^{- \frac{\sigma (1-\alpha )-1}{\sigma -1}}}{\alpha \left( \frac{\sigma }{\sigma -1} \right)^{\alpha } \left( \frac{1}{\sigma F} \right)^{\frac{\alpha }{1-\sigma }}\left( \frac{1-\alpha }{\alpha } \right)^{1-\alpha }}
\end{equation}
\paragraph{Gains from trade.}
$$ \frac{V_n^T}{V_n^A} = \left(\pi_{nn}^T \right)^{- \frac{\alpha }{\sigma -1}} \left( \frac{L_n^T}{L_n^A} \right)^{- \frac{\sigma (1-\alpha )-1}{\sigma -1}} $$
The population share of each location: 
\begin{equation}
  \lambda _n = \frac{L_n}{\bar{L}} = \frac{\left[ A_n^\alpha H_n^{1-\alpha } \pi _{nn}^{-\alpha /(\alpha -1)} \right]^{\frac{\sigma -1}{\sigma (1-\alpha )-1}}}{\sum_{k \in N}^{}\left[ A_k^\alpha H_k^{1-\alpha } \pi _{kk}^{-\alpha /(\alpha -1)} \right]^{\frac{\sigma -1}{\sigma (1-\alpha )-1}}}
\end{equation}
where each location's domestic trade share $\pi _{nn}$ summarizes its market access to other locations.  




\subsection{General Equilibrium}
Two systems of equations across locations. \begin{itemize}
  \item Population mobility.  
  \item Gravity of trade flows
\end{itemize}
\paragraph{Population.}
Using $P_n^\alpha  = \frac{v_n}{\bar{V} r_n^{1-\alpha }}$, $v_n = w_n/\alpha $, and land market clearing ($r_n = \frac{1-\alpha }{\alpha } \frac{w_n L_n}{H_n}$)
\begin{equation}
  P_n = \frac{w_n}{\bar{W}} \left( \frac{H_n}{L_n} \right)^{\frac{1-\alpha }{\alpha }}, \quad \bar{W} \equiv \left[ \alpha \left( \frac{1-\alpha }{\alpha } \right)^{1-\alpha } \bar{V} \right]^{\frac{1}{\alpha }}
\end{equation}
Recall $$ P_n = \frac{\sigma }{\sigma -1} \left( \frac{1}{\sigma F} \right)^{\frac{1}{1-\sigma }} \left[ \sum_{i \in N}^{} L_i \left( d_{ni}  \frac{w_i}{A_i}\right)^{1-\sigma } \right]^{\frac{1}{1-\sigma }} $$
Obtain a first wage equation from population mobility 
\begin{equation}
  \bar{W}^{1-\sigma } \frac{1}{\sigma F} \left( \frac{\sigma }{\sigma -1} \right)^{1-\sigma } = \frac{w_i^{1-\sigma } \left( \frac{H_i}{L_i} \right)^{(1-\sigma ) \frac{1-\alpha }{\alpha }}}{\sum_{n \in N}^{} L_n \left( d_{in} \frac{w_n}{A_n} \right)^{1-\sigma }}\label{eq:sys1}
\end{equation}
% The properties of the general equilibrium of the model can be characterized analytically. 
\paragraph{Gravity.}
Gravity and income equals expenditure implies: 
\begin{equation}
  w_i L_i = \sum_{n \in N}^{} \frac{\frac{L_i}{\sigma F} \left( \frac{\sigma }{\sigma -1} d_{ni} \frac{w_i}{A_i} \right)^{1-\sigma }}{P_n^{1-\sigma }} w_n L_n.
\end{equation}
Recall the price index can be expressed as \eqref{eq:pn}, Obtain a second wage equation from gravity \begin{equation}
  w_i^\sigma A_i^{1-\sigma } = \sum_{n \in N}^{} \pi _{nn} d_{ni}^{1-\sigma } w_n^\sigma A_n^{1-\sigma }.
\end{equation}
Using our expression for the domestic trade share on previous, slide this wage equation from gravity becomes
\begin{equation}
\tilde{W}_{1-\sigma} \frac{1}{\sigma F} \left( \frac{\sigma}{\sigma - 1} \right)^{1-\sigma} = \frac{w_i^{\sigma} A_i^{1-\sigma}}{\sum_{n \in \mathcal{N}} d_{ni}^{1-\sigma} L_n (\sigma - 1)^{1-\alpha} H_n^{(\sigma - 1)^{1-\alpha}} w_n^{\sigma}}\label{eq:sys2}
\end{equation}
Using two systems of equations~\eqref{eq:sys1}~\eqref{eq:sys2} for wages and population yields closed-form solution 
\begin{equation}
  w_n^{1-2\sigma} A_n^{\sigma -1} L_n^{(\sigma -1)\frac{1-\alpha }{\alpha }} H_n^{-(\sigma -1)\frac{1-\alpha }{\alpha }} = \phi .
\end{equation}
After that, this system can be reduced to a single equation that determines the equilibrium population:
\begin{equation}
  L_n ^{\tilde{\sigma }\gamma _1} A_n ^{- \frac{(\sigma -1) (\sigma -1)}{2\sigma - 1}} H_n ^{- \frac{(\sigma -1) (\sigma -1)}{\alpha(2\sigma - 1)}} = \bar{W}^{1-\sigma } \sum_{i \in N}^{} \frac{1}{\sigma F} \left( \frac{\sigma }{\sigma -1} d_{ni} \right)^{1-\sigma } \left( L_i ^{\tilde{\sigma }\gamma _1} \right)^{\frac{\gamma _2}{\gamma _1}} A_i ^{- \frac{\sigma (\sigma -1)}{2\sigma - 1}} H_i ^{\frac{(1-\alpha)(\sigma -1) (\sigma -1)}{\alpha(2\sigma - 1)}}
\end{equation}
where $\bar{W}$ is determined by the requirement that the labor market clears $\sum_{n \in N}^{} L_n = \bar{L}$. 
and $$ \tilde{\sigma } \equiv \frac{\sigma -1}{2 \sigma -1}, \gamma _1 \equiv \sigma \frac{1-\alpha }{\alpha }, \gamma _2 \equiv 1 + \frac{\sigma }{\sigma -1}- (\sigma -1) \frac{1-\alpha }{\alpha } $$

\begin{pros}
Assume $\sigma (1-\alpha )>1$, given the land area, productivity and amenity parameters $\left\{ H_n,A_n,B_n \right\}$ and symmetric bilateral trade frictions $\left\{ d_{ni} \right\}$ for all locations $n,i \in N$, there exist unique equilibrium populations $(L_n^*)$ that solve this system of equations. 
\end{pros}

\subsection{Market access}
Model provides micro-foundations for a theory-consistent measure of market access. 
\begin{itemize}
  \item Ad hoc measures of market potential following Harris (1954): $$ MP _{nt} = \sum_{i \in N}^{} \frac{L_{it}}{dist_{ni}} $$ 
  \item Theory-based measure highlights the role of price indexes. 
\end{itemize}
We now examine the predictions of the model for the equilibrium relationship between wages, population and market access. 

Market access is itself an endogenous variable. 

Combining profit maximization, zero profits, CES demand and market clearing, we obtain the following wage equation:
$$ \left( \frac{\sigma }{\sigma -1} \frac{w_i}{A_i}\right)^\sigma = \frac{1}{\bar{x}_i} FMA_i $$
\begin{equation}
  w_i = \left( \frac{\sigma -1}{\sigma } \right)^{\frac{\sigma -1}{\sigma }} A_i^{\frac{\sigma -1}{\sigma }} (\bar{I})^{-\frac{1}{\sigma }} (FMA_i)^{\frac{1}{\sigma }}
\end{equation}
where firm market access is defined as $$ FMA_i \equiv \sum_{n \in N}^{} d_{ni}^{1-\sigma } (w_n L_n)(P_n)^{\sigma -1} $$
\begin{itemize}
  \item Wages increase in productivity $A_i$ and firm market access $FMA_i$. 
  \item Reductions in transport costs ($d_{ni}$) increase firm market access and wages $w_i$. 
\end{itemize}


\subsection{Model inversion}
Estimate $\sigma $ and $\alpha $ and parameterized symmetric bilateral trade costs $d_{ni}$. 


\subsection{Counterfactual}



\end{document}


